\documentclass[a4paper,12pt]{article}

\usepackage[utf8]{inputenc}
\usepackage[T1]{fontenc}
\usepackage{graphicx}
\usepackage{float}
\usepackage{hyperref}
\usepackage{fullpage}

\author{Antoine CARPENTIER}
\title{Current Trends of Artificial Intelligence\\ \small Programming Assignment : Phase 1}

\bibliographystyle{apalike}

\begin{document}
\maketitle

\section{Introduction}

The project consists of predicting the composer, style, year, key and instrument for several songs, using a dataset from the Jazzomat Research Project (\url{http://jazzomat.hfm-weimar.de/dbformat/dbcontent.html}).

\section{Computational representation : features vector}

We decided to use feature vector based methods. We are presenting here features that we might use. This list is not exhaustive. The features are classed by the charateristic to predict.

\subsection{Note}

Notes are charaterized by duration, pitch and velocity. Because there might be a lot of notes, we propose to use the following features to reduce the size of the feature vector.

\begin{itemize}
    \item Mean and standard deviation of duration (long notes or short notes)
    \item Standard deviation of pitch of the notes (monotone or not)
    \item Mean and standard deviation of velocity (sharp notes or not)
    \item Mean and standard deviation of the note/time distribution (mean number of notes by unit of time + "burst of notes" =~ rythm of the song) 
\end{itemize}

\subsection{Key}

Key depends on the time signature and the notes. 

\begin{itemize}
    \item statistical frequency of each note \todo{other ??}
    % depending on the time signature, the sequence of the first note of bars
\end{itemize}

\subsection{Composer}

Composer depends on the features of note, key and time signature

\subsection{Style}

Style depends on the features of notes, key and time signature

\subsection{Year}

Year depends on style and composer

\subsection{Instrument}

Instrument depends on year, style and composer

\bibliography{main}
\end{document}
