\documentclass[a4paper,12pt]{article}

\usepackage[utf8]{inputenc}
\usepackage[T1]{fontenc}
\usepackage{graphicx}
\usepackage{float}
\usepackage{hyperref}
\usepackage{fullpage}
\usepackage{todo}

\author{\textbf{DiscoveryTripToMusic}\\CARPENTIER Antoine - 0529527 \\LIU Xinyu - 0526434}
\title{Current Trends of Artificial Intelligence\\ \small Programming Assignment : Phase 1}

\bibliographystyle{apalike}

\begin{document}
\maketitle

\section{Introduction}

The project consists of predicting the composer, style, year, key and instrument for several songs, using a dataset from the Jazzomat Research Project (\url{http://jazzomat.hfm-weimar.de/dbformat/dbcontent.html}).

\section{Computational representation : features vector}

We decided to use feature vector based methods. We are presenting here features that we might use. This list is not exhaustive as we might add or remove several features to improve performance.

\subsection{Features}
    Notes are charaterized by duration, pitch and velocity. For each song, we have also have tempo and time signature data. Some songs also have a key data. Because the key data is not always present, we might use semi-supervised learning for the key prediction.
    Other predictions will use the following features derived from notes, tempo and time signature. Most features are inspired from \cite{mckay2004} and already used in \cite{mckay2004automatic} and \cite{mckay2004automatic2}.

    \subsubsection{Pitch-based features}
        \begin{itemize}
            \item Difference between the highest and lowest pitch
            \item Mean and standard deviation of the pitch : this will indicate the monotony of the song
            \item The proportion of bass, medium and high pitches (the three classes are defined as MIDI notes respectively from 0 to 53, 54 to 71 and 72 to 127)
            \item Most common chord used. A \textbf{chord} is defined as a group of notes played at the same time
            \item Mean interval between the most common chord
        \end{itemize}
    \subsubsection{Duration-based features}
        \begin{itemize}
            \item Difference between the highest and lowest duration
            \item Mean and standard deviation of duration
            \item Maximum and minimum note duration
            \item Mean and standard deviation of interval between notes (silences)
        \end{itemize}

    \subsubsection{Velocity-based features}
        \begin{itemize}
            \item Difference between the highest and lowest velocity
            \item Mean and standard deviation of velocity. This will allow to measure the general volume of the song and the emphasis given on notes.
            \item Proportion of "strong" notes (notes with the highest velocity)
        \end{itemize}

    \subsubsection{Note density based features}
        We will compute the distribution of notes by unit of time (for example five seconds). \textbf{Phrases} are defined as clusters of notes.
    \begin{itemize}     
        \item Mean and standard deviation of the note density. This will allow to measure if there are a lot of "burst of notes" or if the notes are regularly distributed. It will give an insight at the rythm of the song.
        \item Mean and standard deviation of the length of a phrase
        \item Number of phrases
    \end{itemize}

\subsection{Further improvements}

We present here improvements that could be made to the implementation of features.

\begin{itemize}
    \item Extensibility: The first thing we would like to do is to permit extensibility, since it will help us to support new concepts and structures. Changing the parameters is the key for improving the performance of our representation.
    \item We propose to change the classification to be automatical by using a hierarchy and updating the feature extraction in real time. That way we could select the best features.
    \item If the further data is sufficient enough, we would like to try different structures to present our features or we would like to preprocess our data with some mathematic models. This could be easily done using the raw audio signal for each song. Furthermore, we also would like to consider some temporal behavior of features, like psychological influence, temporal envelope processing by the human auditory system, if there are relative data samples. Because some research propose that the temporal behavior of features is important for music classification \cite{mckinney2003features}.
\end{itemize}

\bibliography{main}
\end{document}
