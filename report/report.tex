\documentclass[a4paper,12pt]{article}

\usepackage[utf8]{inputenc}
\usepackage[T1]{fontenc}
\usepackage{graphicx}
\usepackage{float}
\usepackage{hyperref}
\usepackage{fullpage}
\usepackage{todo}

\author{Antoine CARPENTIER}
\title{Current Trends of Artificial Intelligence\\ \small Programming Assignment : Phase 1}

\bibliographystyle{apalike}

\begin{document}
\maketitle

\section{Introduction}

The project consists of predicting the composer, style, year, key and instrument for several songs, using a dataset from the Jazzomat Research Project (\url{http://jazzomat.hfm-weimar.de/dbformat/dbcontent.html}).

\section{Computational representation : features vector}

We decided to use feature vector based methods. We are presenting here features that we might use. This list is not exhaustive. The features are classed by the charateristic to predict.

\subsection{Features}
    Notes are charaterized by duration, pitch and velocity. For each song, we have also have tempo and time signature data. Some songs also have a key data. Because the key data is not always present, we might use semi-supervised learning for the key prediction.
    Other predictions will use the following features derived from notes, tempo and time signature : 

    \begin{itemize}
        \item Difference between the highest and lowest pitch
        \item Difference between the highest and lowest duration
        \item Difference between the highest and lowest velocity
        \item Mean and standard deviation of duration (long notes or short notes)
        \item Mean and standard deviation of pitch of the notes (monotone or not)
        \item Mean and standard deviation of velocity (general volume and emphasis given on each note)
        \item Mean and standard deviation of the note density (mean number of notes by unit of time + "burst of notes" =~ rythm of the song) 
        \item Mean length of a phrase
        \item Number of phrases
        \item Mean and standard deviation of interval between notes
        \item statistical frequency of each note \todo{other ??}
        % depending on the time signature, the sequence of the first note of bars
    \end{itemize}

\subsection{Further improvements}

We present here improvements that could be made to the implementation of features.

\begin{itemize}
    \item Extensibility: The first thing we would like to do is to permit extensibility, since it will help us to support new concepts and structures. Changing the parameters is the key for improving the performance of our representation.
    \item We propose to change the classification to be automatical by using a hierarchy and updating the feature extraction in real time. That way we could select the best features.
    \item If the further data is sufficient enough, we would like to try different structures to present our features or we would like to preprocess our data with some mathematic models. This could be easily done using the raw audio signal for each song. Furthermore, we also would like to consider some temporal behavior of features, like psychological influence, temporal envelope processing by the human auditory system, if there are relative data samples. Because some research propose that the temporal behavior of features is important for music classification \cite{mckinney2003features}.
\end{itemize}

\bibliography{main}
\end{document}
