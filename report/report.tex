\documentclass[a4paper,12pt]{article}

\usepackage[utf8]{inputenc}
\usepackage[T1]{fontenc}
\usepackage{graphicx}
\usepackage{float}
\usepackage{hyperref}
\usepackage{fullpage}

\author{Antoine CARPENTIER}
\title{Current Trends of Artificial Intelligence\\ \small Programming Assignment : Phase 1}

\bibliographystyle{apalike}

\begin{document}
\maketitle

\section{Introduction}

The project consists of predicting the composer, style, year, key and instrument for several songs, using a dataset from the Jazzomat Research Project (\url{http://jazzomat.hfm-weimar.de/dbformat/dbcontent.html}).

\section{Computational representation : features vector}

We decided to use feature vector based methods. We are presenting here features that we might use. This list is not exhaustive. The features are classed by the charateristic to predict.

\subsection{Key}

Key depends on the time signature and the notes. Notes are charaterized by duration, pitch and velocity. Because there might be a lot of notes, we propose to use the following features :

\begin{itemize}
    \item Mean of duration of the notes
    \item Standard deviation of pitch of the notes
    \item Mean and standard deviation of velocity
    \item The set mel-frequency cepstral coefficients (MFCC) for music discrimination tasks.
    \item
    
    
    
    \item Further research 1: Automatic classification(using a hierarchy that could be represented as a tree and the feature
    \item extraction should update in real time)
    \item Further research 2: Temporal behavior of features, such as Psychoacoustic features, Auditory filterbank temporal       \item envelopes.
    
\end{itemize}

\subsection{Composer}

\subsection{Style}

\subsection{Year}

\subsection{Instrument}

\bibliography{main}
\end{document}
